\documentclass{article}
\usepackage{donow}

\begin{document}

\centerline{\textbf{Euler 145}}

For this problem, we can write down a closed form solution for $R(n)$, the number of reversible numbers with $n$ digits. The main observations come from keeping track of carries. By propogating the information that there is no incoming carry to the 1s place in the reverse-sum, it is easy to see that if $n$ is even, then there can be no carries in the addition. Now, counting shows that there are 30 ways to assign a pair of digits so that the result is an odd number with no carries, and 20 ways to assign so that the leading digit isn't 0. So, if $n$ is even, $R(n) = 30^{\frac{n}2 - 1}\cdot20$. If $n$ is odd, then the middle digit requires a carry to become odd, and propogating this information gives us that there must be alternating carry and non-carries in the digits. The number of ways of assigning so there is a carry out within a carry in is 20, the number of choices with a carry in but no out is 25. Also, this works if $N \equiv 3 \mod 4$, but if $N \equiv 1 \mod 4$, then there is actually no way of finding a reversible number. Also, there are 54 ways to choose the middle digit. Combining all of this information, we find that

\[R(n) = \begin{cases}
  30^{n/2 - 1} & n \text{ even}\\
  5\cdot20^{\ceil{\frac{n-1}4}}25^{\floor{\frac{n-1}4}} & n\equiv 3 \mod 4\\
  0 & n\equiv 1 \mod 4
  \end{cases}
\]
Also we note that $R(0) = R(1) = 0$. Using this, the solution to this problem is
\[\sum_{n=2}^8 R(n) = 20 + 20\cdot30 + 20\cdot30^2 + 20\cdot30^3 + 5\cdot20 + 5\cdot20^2\cdot25 = 608720\]


\end{document}
